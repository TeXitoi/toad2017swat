\documentclass[table]{beamer}
%%\documentclass[handout]{beamer}

\mode<presentation>
{
  \definecolor{navitialight}{RGB}{126,186,200}
  \definecolor{navitiadark}{RGB}{76,102,114}

  \useoutertheme[subsection=false]{miniframes}
  %%\useoutertheme[footline=authortitle]{miniframes}
  \usecolortheme[named=navitiadark]{structure}
  %%\usecolortheme{dolphin}
  \usecolortheme{orchid}
  \useinnertheme{circles}
  \setbeamerfont{block title}{size=\normalsize}
  \setbeamercovered{transparent}

  %%% le foot pour avoir la numérotation des slides %%%
  \setbeamertemplate{footline}{%
    \leavevmode%
    \hbox{%
      \begin{beamercolorbox}[wd=.5\paperwidth,ht=2.5ex,dp=1.125ex,
        leftskip=.3cm plus1fill,rightskip=.3cm]{author in head/foot}%
        \usebeamerfont{title in head/foot}\insertshorttitle
      \end{beamercolorbox}%
      \begin{beamercolorbox}[wd=.5\paperwidth,ht=2.5ex,dp=1.125ex,
        leftskip=.3cm,rightskip=.3cm plus1fil]{title in head/foot}%
        \usebeamerfont{author in head/foot}\insertshortauthor\hfill
        \insertframenumber/\inserttotalframenumber
      \end{beamercolorbox}%
    }%
    \vskip0pt%
  }

  \setbeamercolor{palette primary}{fg=white,bg=navitiadark}
  \setbeamercolor{palette secondary}{fg=white,bg=navitialight}
  \setbeamercolor{palette tertiary}{fg=white,bg=navitiadark}
  \setbeamercolor{palette quaternary}{fg=white,bg=navitialight}
}

\mode<handout>
{
  \usepackage{pgfpages}
  \pgfpagesuselayout{4 on 1}[a4paper,border shrink=5mm,landscape]
}

\usepackage[utf8]{inputenc}
\usepackage{lmodern}
\usepackage[T1]{fontenc}
\usepackage[english,francais]{babel}
\usepackage{multirow}
\usepackage{hhline}

\newcommand{\nologo}{\setbeamertemplate{logo}{}}

\newenvironment{foreignpar}[1][english]{%
    \em\selectlanguage{#1}%
}{}
\newcommand*{\foreign}[2][english]{%
    \emph{\foreignlanguage{#1}{#2}}%
}

\title{Transport Optimisé À la Demande}

\author{L'équipe TOÀD}
%%\author{Vincent Chabot \and Romain Choquet \and Guillaume Pinot \and Pascal Rhod}

\institute[Kisio Digital] % (optional, but mostly needed)
{
  Kisio Digital\\
  20 rue Hector Malot\\
  75012 Paris, France}
%% - Use the \inst command only if there are several affiliations.
%% - Keep it simple, no one is interested in your street address.

\date{Soutenance SWAT 2017}
%% - Either use conference name or its abbreviation.
%% - Not really informative to the audience, more for people (including
%%   yourself) who are reading the slides online

%% If you have a file called "university-logo-filename.xxx", where xxx
%% is a graphic format that can be processed by latex or pdflatex,
%% resp., then you can add a logo as follows:
%\pgfdeclareimage[width=.2\linewidth]{logo}{images/logo_nio}
%\logo{\pgfuseimage{logo}\hspace{.04\linewidth}}


%% Delete this, if you do not want the table of contents to pop up at
%% the beginning of each subsection:
\AtBeginSection[]
{
  \begin{frame}<beamer>
    \frametitle{Table des matières}
    \tableofcontents[currentsection,hideothersubsections]
  \end{frame}
}
\AtBeginSubsection[]
{
  \begin{frame}<beamer>
    \frametitle{Table des matières}
    \tableofcontents[currentsection,subsectionstyle=show/shaded/hide]
  \end{frame}
}


\begin{document}

\begin{frame}
  \titlepage
\end{frame}

\section{Problématique}

\begin{frame}
  \frametitle{Problématique}

  Problème du transport en commun en demande faible:
  \begin{itemize}
  \item mauvais service pour le voyageur
  \item coût élevé pour le transporteur
  \end{itemize}

  Notre but:
  \begin{itemize}
  \item améliorer la qualité de service pour le voyageur
  \item optimiser les ressources pour les transporteurs
  \end{itemize}

  Notre moyen d'y parvenir:
  \begin{itemize}
  \item répondre en temps réel aux demandes de voyageur
  \item transporter le voyageur où il demande
  \item mutualiser les courses sur les véhicules
  \item être flexible sur les véhicules utilisés.
  \end{itemize}
\end{frame}

\section{Simulation}

\begin{frame}
  \frametitle{Vision voyageur}

  cf exemple
\end{frame}

\begin{frame}
  \frametitle{Vision conducteur}
\end{frame}

\begin{frame}
  \frametitle{Comment ça marche: les grandes idées}
\end{frame}

\section{Business Model}

\begin{frame}
  \frametitle{Business Model}

  Nos clients: les transporteurs
  \begin{itemize}
  \item orthogonal au marché saturé des VTC
  \item les clients actuels de Kisio sont de potentiel clients
  \item nos futurs clients sont des clients potentiel de Kisio
  \end{itemize}
\end{frame}

\section{La concurrence}

\begin{frame}
  \frametitle{La concurrence}

  \begin{itemize}
  \item uber pool, allygator shuttle...: solution orthogonal sur le
    principe d'uber
  \item padam: exactement pareil?
  \item bestmile: pareil, spécialisé véhicule autonome
  \end{itemize}
\end{frame}

\section{Roadmap}

\begin{frame}
  \frametitle{Roadmap}
  Avant le SWAT:
  \begin{itemize}
  \item Prise de contact avec des transporteur dans l'optique d'avoir
    un jeu de donnée réaliste pour simulation
  \end{itemize}

  Pendant le SWAT, en parallèle et incrémental
  \begin{itemize}
  \item Création/amélioration/analyse du jeu de donnée de simulation
  \item Création du moteur de génération de tournée
  \item Création de l'API
  \item Création des IHM voyageurs, conducteurs et supervision
  \end{itemize}
  
  Après le SWAT
  \begin{itemize}
  \item présentation aux clients potentiel
  \item simulations de cas réels de ces clients
  \item expérimentations
  \item ...
  \end{itemize}
\end{frame}

\section{Conclusion}

\begin{frame}
  \frametitle{Conclusion}
\end{frame}

\begin{frame}
  \titlepage
\end{frame}

\end{document}
