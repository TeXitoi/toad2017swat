\documentclass[table]{beamer}
%%\documentclass[handout]{beamer}

\mode<presentation>
{
  \definecolor{navitialight}{RGB}{126,186,200}
  \definecolor{navitiadark}{RGB}{76,102,114}

  \useoutertheme[subsection=false]{miniframes}
  %%\useoutertheme[footline=authortitle]{miniframes}
  \usecolortheme[named=navitiadark]{structure}
  %%\usecolortheme{dolphin}
  \usecolortheme{orchid}
  \useinnertheme{circles}
  \setbeamerfont{block title}{size=\normalsize}
  \setbeamercovered{transparent}

  %%% le foot pour avoir la numérotation des slides %%%
  \setbeamertemplate{footline}{%
    \leavevmode%
    \hbox{%
      \begin{beamercolorbox}[wd=.5\paperwidth,ht=2.5ex,dp=1.125ex,
        leftskip=.3cm plus1fill,rightskip=.3cm]{author in head/foot}%
        \usebeamerfont{title in head/foot}\insertshorttitle
      \end{beamercolorbox}%
      \begin{beamercolorbox}[wd=.5\paperwidth,ht=2.5ex,dp=1.125ex,
        leftskip=.3cm,rightskip=.3cm plus1fil]{title in head/foot}%
        \usebeamerfont{author in head/foot}\insertshortauthor\hfill
        \insertframenumber/\inserttotalframenumber
      \end{beamercolorbox}%
    }%
    \vskip0pt%
  }

  \setbeamercolor{palette primary}{fg=white,bg=navitiadark}
  \setbeamercolor{palette secondary}{fg=white,bg=navitialight}
  \setbeamercolor{palette tertiary}{fg=white,bg=navitiadark}
  \setbeamercolor{palette quaternary}{fg=white,bg=navitialight}
}

\mode<handout>
{
  \usepackage{pgfpages}
  \pgfpagesuselayout{4 on 1}[a4paper,border shrink=5mm,landscape]
}

\usepackage[utf8]{inputenc}
\usepackage{lmodern}
\usepackage[T1]{fontenc}
\usepackage[english,francais]{babel}
\usepackage{multirow}
\usepackage{hhline}

\newcommand{\nologo}{\setbeamertemplate{logo}{}}

\newenvironment{foreignpar}[1][english]{%
    \em\selectlanguage{#1}%
}{}
\newcommand*{\foreign}[2][english]{%
    \emph{\foreignlanguage{#1}{#2}}%
}

\title{Transport Optimisé À la Demande}

\author{L'équipe TOÀD}
%%\author{Vincent Chabot \and Romain Choquet \and Guillaume Pinot \and Pascal Rhod}

\institute[Kisio Digital] % (optional, but mostly needed)
{
  Kisio Digital\\
  20 rue Hector Malot\\
  75012 Paris, France}
%% - Use the \inst command only if there are several affiliations.
%% - Keep it simple, no one is interested in your street address.

\date{Soutenance SWAT 2017}
%% - Either use conference name or its abbreviation.
%% - Not really informative to the audience, more for people (including
%%   yourself) who are reading the slides online

%% If you have a file called "university-logo-filename.xxx", where xxx
%% is a graphic format that can be processed by latex or pdflatex,
%% resp., then you can add a logo as follows:
%\pgfdeclareimage[width=.2\linewidth]{logo}{images/logo_nio}
%\logo{\pgfuseimage{logo}\hspace{.04\linewidth}}


%% Delete this, if you do not want the table of contents to pop up at
%% the beginning of each subsection:
\AtBeginSection[]
{
  \begin{frame}<beamer>
    \frametitle{Table des matières}
    \tableofcontents[currentsection,hideothersubsections]
  \end{frame}
}
\AtBeginSubsection[]
{
  \begin{frame}<beamer>
    \frametitle{Table des matières}
    \tableofcontents[currentsection,subsectionstyle=show/shaded/hide]
  \end{frame}
}


\begin{document}

\begin{frame}
  \titlepage
\end{frame}

\section{Problématique}

\begin{frame}{Problématique}

  Problème du transport en commun en demande faible:
  \begin{itemize}
  \item mauvais service pour le voyageur
  \item coût élevé pour le transporteur
  \end{itemize}

  Notre but:
  \begin{itemize}
  \item améliorer la qualité de service pour le voyageur
  \item optimiser les ressources pour les transporteurs
  \end{itemize}

  Notre moyen d'y parvenir:
  \begin{itemize}
  \item répondre en temps réel aux demandes de voyageur
  \item transporter le voyageur où il le souhaite
  \item mutualiser les courses sans pénaliser le temps de trajet
  \item être flexible sur les types de véhicule utilisés
  \end{itemize}
\end{frame}

\section{Simulation}

\begin{frame}{Vision voyageur}

  \begin{itemize}[<+->]
  \item Alice veut aller du 20 rue Hector Malot à la Tour Eiffel, elle
    le saisie dans l'interface. Il est 03h37.
  \item L'IHM lui propose de monter dans le transport au 44 boulevard
    Diderot à 03h45, et qu'elle arrivera au 3 avenue de Suffren avant
    04h45.
  \item Alice accepte la proposition, elle est à l'heure au point de
    rendez-vous.
  \item Le véhicule arrive à 03h52. Bob est déjà dans le véhicule.
  \item À 04h18, Bob descend 108 boulevard de Grenelle.
  \item À 04h20, Cunégonde monte au 129 boulevard de Grenelle.
  \item Alice descend à 04h31 au 3 avenue de Suffren.
  \end{itemize}
\end{frame}

\begin{frame}[allowframebreaks]{Vision conducteur}
  Le conducteur a une liste d'étape à réaliser avec les
  montées/descentes, se modifiant en temps réel:\framebreak

  À 03h17:
  \begin{itemize}
  \item 03h34, 158 rue de Bagnolet, Bob monte
  \item 04h13, 108 rue de Grenelle, Bob descend
  \end{itemize}\framebreak

  À 03h34, Bob monte:
  \begin{itemize}
  \item<42> 03h34, 158 rue de Bagnolet, Bob monte
  \item 04h13, 108 rue de Grenelle, Bob descend
  \end{itemize}\framebreak

  À 03h37, Alice fait sa réservation:
  \begin{itemize}
  \item \textbf{03h52, 44 boulevard Diderot, Alice monte}
  \item \textbf{04h18,} 108 rue Grenelle, Bob descend
  \item \textbf{04h25, 3 avenue de Suffren, Alice descend}
  \end{itemize}\framebreak

  À 03h52, Alice monte:
  \begin{itemize}
  \item<42> 03h52, 44 boulevard Diderot, Alice monte
  \item 04h18, 108 rue Grenelle, Bob descend
  \item 04h25, 3 avenue de Suffren, Alice descend
  \end{itemize}\framebreak

  À 04h07, Cunégonde fait sa réservation:
  \begin{itemize}
  \item 04h18, 108 boulevard Grenelle, Bob descend
  \item \textbf{04h20, 129 boulevard de Grenelle, Cunégonde monte}
  \item \textbf{04h31,} 3 avenue de Suffren, Alice descend
  \item \textbf{05h14, 3 Avenue Gambetta, Cunégonde descend}
  \end{itemize}\framebreak

  À 04h18, Bob descend:
  \begin{itemize}
  \item<42> 04h18, 108 boulevard Grenelle, Bob descend
  \item 04h20, 129 boulevard de Grenelle, Cunégonde monte
  \item 04h31, 3 avenue de Suffren, Alice descend
  \item 05h14, 3 Avenue Gambetta, Cunégonde descend
  \end{itemize}\framebreak

  À 04h20, Cunégonde monte:
  \begin{itemize}
  \item<42> 04h20, 129 boulevard de Grenelle, Cunégonde monte
  \item 04h31, 3 avenue de Suffren, Alice descend
  \item 05h14, 3 Avenue Gambetta, Cunégonde descend
  \end{itemize}\framebreak

  À 04h31, Alice descend:
  \begin{itemize}
  \item<42> 04h31, 3 avenue de Suffren, Alice descend
  \item 05h14, 3 Avenue Gambetta, Cunégonde descend
  \end{itemize}\framebreak

  À 05h14, Cunégonde descend:
  \begin{itemize}
  \item<42> 05h14, 3 Avenue Gambetta, Cunégonde descend
  \end{itemize}
\end{frame}

\begin{frame}{Comment ça marche: les grandes idées}
  \begin{itemize}
  \item une requête de réservation est de la forme \emph{heure de
      départ, origine, destination}
  \item lors de la réception d'une requête, le système calcule une heure
    d'arrivée au plus tard en fonction du temps de trajet direct
  \item le moteur vérifie la possibilité de satisfaire la demande
  \item si c'est possible, la demande est proposée
  \item à l'acceptation de la demande, elle devient une réservation
  \item à tout moment, les tournées peuvent être modifiées pour
    améliorer le plan de transport.
  \end{itemize}
\end{frame}

\section{Business Model}

\begin{frame}{Business Model}

  Nos clients: les transporteurs
  \begin{itemize}
  \item orthogonal au marché saturé des VTC
  \item les clients actuels de Kisio sont des clients potentiels
  \item les clients de TOAD sont des clients potentiels pour Kisio
  \end{itemize}
\end{frame}

\section{La concurrence}

\begin{frame}{La concurrence}

  \begin{itemize}
  \item uber pool, allygator shuttle...: solution orthogonal sur le
    modèle classique VTC
  \item padam: après un essai avec sa propre flotte, revente de la
    solution logiciel
  \item bestmile: spécialisé véhicule autonome, cible les
    transporteurs, sites privés, opérateurs de taxi, constructeurs de
    véhicules
  \item via: partenariat avec LeCab sur Paris, principalement modèle
    classique VTC mais démarche également les transporteurs
  \end{itemize}
\end{frame}

\section{Roadmap}

\begin{frame}{Roadmap}
  Avant le SWAT:
  \begin{itemize}
  \item prise de contact avec des transporteur dans l'optique d'avoir
    un jeu de donnée réaliste pour simulation
  \end{itemize}

  Pendant le SWAT, en parallèle et incrémental
  \begin{itemize}
  \item création/amélioration/analyse du jeu de donnée de simulation
  \item création du moteur de génération de tournée
  \item création de l'API
  \item création des IHM voyageurs, conducteurs et supervision
  \end{itemize}
  
  Après le SWAT
  \begin{itemize}
  \item présentation aux clients potentiel
  \item simulations de cas réels de ces clients
  \item expérimentations
  \item ...
  \end{itemize}
\end{frame}

\section{Conclusion}

\begin{frame}{Conclusion}
  Rendre le \textbf{transport en commun en demande faible} (zone peu
  dense, heures creuses, service de nuit) \textbf{plus efficace} pour
  le \textbf{voyageur} (qualité de service) et le
  \textbf{transporteur} (coût).
\end{frame}

\begin{frame}
  \titlepage
\end{frame}

\end{document}
